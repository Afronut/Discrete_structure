\documentclass[12pt,letterpaper]{article}
\usepackage{fullpage}
\usepackage[top=2cm, bottom=4.5cm, left=2.5cm, right=2.5cm]{geometry}
\usepackage{amsmath,amsthm,amsfonts,amssymb,amscd}
\usepackage{lastpage}
\usepackage{enumerate}
\usepackage{fancyhdr}
\usepackage{mathrsfs}
\usepackage{xcolor}
\usepackage{graphicx}
\usepackage{listings}
\usepackage{hyperref}
\usepackage{booktabs}
\usepackage{array}
\usepackage{xcolor}
\usepackage{colortbl}
\usepackage{amsmath}

\hypersetup{%
  colorlinks=true,
  linkcolor=blue,
  linkbordercolor={0 0 1}
}
\renewcommand\lstlistingname{Algorithm}
\renewcommand\lstlistlistingname{Algorithms}
\def\lstlistingautorefname{Alg.}

\lstdefinestyle{Python}{
    language        = Python,
    frame           = lines, 
    basicstyle      = \footnotesize,
    keywordstyle    = \color{blue},
    stringstyle     = \color{green},
    commentstyle    = \color{red}\ttfamily
}

\setlength{\parindent}{0.0in}
\setlength{\parskip}{0.05in}

% Edit these as appropriate
\newcommand\course{CSCI 2824}
\newcommand\hwnumber{8}                  % <-- homework number
\newcommand\NetIDa{Coovi Meha}           % <-- NetID of person #1
\newcommand\NetIDb{SID:107069433}           % <-- NetID of person #2 (Comment this line out for problem sets)

\pagestyle{fancyplain}
\headheight 35pt
\lhead{\NetIDa}
\lhead{\NetIDa\\\NetIDb}                 % <-- Comment this line out for problem sets (make sure you are person #1)
\chead{\textbf{\Large Homework \hwnumber}}
\rhead{\course \\ \today}
\lfoot{}
\cfoot{}
\rfoot{\small\thepage}
\headsep 1.5em

\begin{document}

\section*{Problem 1}
Use induction to prove that the following identities hold for all n\(\geq\)1
\begin{itemize}
      \item \(\sum_{n=1}^{n} i^2=\frac{n(n+1)(2n+1)}{6}\)\\
      Let p(n)  be the preposition that \(\sum_{n=1}^{n} i^2=\frac{n(n+1)(2n+1)}{6}\) is true\\ 
      \textbf{Basic Step:} p(1) is true, because when n=1 \(1^2\)=\(\frac{1(1+1)(2+1)}{6}\)=1 is true\\
      \textbf{Inductive Hypothesis:} The induction hypothesis is the statement
      that p(k) is true, that is, \(\sum_{n=1}^{n} i^2=\frac{n(n+1)(2n+1)}{6}\) where k is an arbitrary nonnegative integer
      greater than 1. We must show that if P (k) is true,then P (k+1), which states that
      \(\sum_{i=1}^{K} (i+1)^2=\frac{(k+1)(k+2)(2(k+1)+1)}{6}=\frac{(k+1)(k+2)(2k+3)}{6}\)\\
      \(1^2+2^2+...+k^2+(K+1)^2 \)=\((1^2+2^2+...+k^2)+(K+1)^2 \)\\
      from our hypothesis we have \\
      \(\frac{k(k+1)(2k+1)}{6}+(K+1)^2\)\\
      \((k+1)\frac{k(2k+1)}{6}+(K+1)\)\\
      \((k+1)\frac{k(2k+1)+6(K+1)}{6}\)\\
      \(\frac{(k+1)(2k+3)(K+2)}{6}\)
      \item \(\sum_{n=1}^{n} i^3=\frac{n^2(n+1)^2}{4}\)\\
      Let p(n)  be the preposition that \(\sum_{n=1}^{n} i^3=\frac{n^2(n+1)^2}{4}\) is true\\
      \textbf{Basic Step:} p(1) is true, because when n=1 \(1^3=\frac{1^2(1+1)^2}{4}\)=1 is true\\
      \textbf{Inductive Hypothesis:} The induction hypothesis is the statement
      that p(k) is true, that is, \(\sum_{n=1}^{n} i^3=\frac{n^2(n+1)^2}{4}\) where k is an arbitrary nonnegative integer
      greater than 1. We must show that if P (k) is true,then P (k+1), which states that
      \(\sum_{i=1}^{K} (i+1)^2=\frac{(k+1)^2(k+2)^2}{4}\)\\
      \(1^3+2^3+...+k^3+(K+1)^3 \)=\((1^3+2^3+...+k^3)+(K+1)^3 \)\\
      from our hypothesis we have \\
      \(\frac{k^2(k+1)^2}{4} +(K+1)^3=\frac{(k+1)^2(k+2)^2}{4}\)
      \item \((1-\frac{1}{4})(1-\frac{1}{9})...(1-\frac{1}{n^2})=\frac{n+1}{2n}\)\\
      Let p(n)  be the preposition that \((1-\frac{1}{4})(1-\frac{1}{9})...(1-\frac{1}{n^2})=\frac{n+1}{2n}\)\\
      \textbf{Basic Step:} p(1) is true, because when n=2 \(\frac{3}{4}=\frac{3}{4}\) is true\\
      \textbf{Inductive Hypothesis:} The induction hypothesis is the statement
      that p(k) is true, that is \((1-\frac{1}{4})(1-\frac{1}{9})...(1-\frac{1}{n^2})=\frac{n+1}{2n}\),  where k is an arbitrary nonnegative integer
      greater than 2. We must show that if P (k) is true,then P (k+1), which states that\\
      \((1-\frac{1}{4})(1-\frac{1}{9})...(1-\frac{k+1}{2k})(1-\frac{1}{(k+1)^2})=\frac{k+2}{2(k+1)}\)
      from our hypothesis we have \\
      \((\frac{k+1}{2k})(1-\frac{1}{(k+1)^2})\)=\(\frac{k+2}{2(k+1)}\)
      \section*{Problem 2}
      Let the sequence Tn be defined by T1=T2=T3= 1 andTn=Tn−1+Tn−2+Tn−3 for n\(\geq\)4.  Use induction to prove that Tn<\(2^n\)for n\(\geq\)4\\
      using strong induction\\
      Let p(n)  be the preposition that Tn<\(2^n\)\\
      \textbf{Basic Step:} p(4) is true, because when n=4, T4=3<16\\
      T1=T2=T3\(\leq\)2
      \textbf{Inductive Hypothesis:} The induction hypothesis is the statement
      that p(k) is true, that is Tk<\(2^k\),  where k is an arbitrary nonnegative integer
      greater than 4. We must show that if P (k) is true,then P (k+1), which states that T(k+1)<\(2^{K+1}\)\\
      T(k+1)=T(k)+T(k-1)+T(k-2)\\=\(2^k+2^{k-1}+2^{k-2}\)\(\leq\)\(2^{k-2+2}+2^{k-2+1}+2^{k-2}\)\\
      \(\leq\)\(2^k(4+1+2)\)\\
      \(\leq\)\(2^k\)

      \section*{Problem 3}
      Consider the function \(f(n) = 50n^3+6n^3log(n^3)-nlog(n^2)\) which represents the complexity of somealgorithm
      \begin{enumerate}
            \item Find a tight big-O bound of the form g(n) =\(n^p\) for the given function f with some natural number p.  What are the constants C and k from the big-O definition?\\
            \(f(n) = 50n^3+6n^3log(n^3)-nlog(n^2)\)\\
            \(50n^3 \leq n^4\)\\
            \(6n^3log(n^3)=18n^3log(n)\leq 18n^3*n \leq 18n^4\)\\
            \(nlog(n^2)\leq 2nlog(n) \leq 2n*n \leq 2n^2 \leq 2n^4\) for n \(\geq\) 1\\ 
            so f is a \(\mathbb{O}(n^4)\) with C=21 and k=1
            \item Find a tight big-\(\Omega\) bound of the form g(n)=\(n^p\) for the given function f with some natural number p.  What are the constants C and k from the big-\(\Omega\) definition?\\
            \(f(n) = 50n^3+6n^3log(n^3)-nlog(n^2)\)\\
            \(50n^3 \geq n^3\)\\
            \(6n^3log(n^3)=18n^3log(n)\geq 18n^3 for n \geq e\)\\
            \(nlog(n^2)\leq 2nlog(n) \geq 0\) for n \(\geq\) 1\\
            so f is a \(\Omega\)(\(n^3)\) with C=19 and k=6
            \item  Since \(\Omega\)(\(n^3)\) and \(\mathbb{O}(n^4)\) do not have the same 
            p, we can not conclude that \(\ominus\)(\(n^p\))
      \end{enumerate}
      \section*{Problem 4}
      Multiply the following matrices:
      \begin{enumerate}
            \item \[
                  \begin{bmatrix}
                        3 & 4\\
                        1 & 0\\
                        2 & 7
                      \end{bmatrix}*
                  \begin{bmatrix}
                  8 & 1 & 2\\
                  7 & 6 & 5
                  \end{bmatrix}\\
                  =\begin{bmatrix}
                        3*8+7*4 & 3*1+4*6 & 3*2+ 4*5\\
                        1*8+0*7 & 1*1+0*6 & 1*2+ 0*5\\
                        2*8+7*7 & 1*2+6*7 & 2*2+ 7*5
                  \end{bmatrix}
                  =\begin{bmatrix}
                        52 & 27 & 24\\
                        8 & 1 & 2\\
                        65 & 44 & 39
                  \end{bmatrix}
                  \] 
            \item \[
                  \begin{bmatrix}
                        a_{11} & a_{12} & a_{13}\\
                        a_{21} & a_{22} & a_{23}\\
                        a_{31} & a_{32} & a_{33}
                      \end{bmatrix}*
                      \begin{bmatrix}
                        b_{11} & b_{12} & b_{13}\\
                        b_{21} & b_{22} & b_{23}\\
                        b_{31} & b_{32} & b_{33}
                      \end{bmatrix}=
                  \] 
                  \[
                  \begin{bmatrix}
                        a_{11}*b_{11} + a_{12}*b_{21} + a_{13}*b_{31} & a_{21}*b_{11} + a_{22}*b_{21} + a_{23}*b_{31} & a_{31}*b_{11} + a_{32}*b_{21} + a_{33}*b_{31}\\
                        a_{11}*b_{12} + a_{12}*b_{22} + a_{13}*b_{32} & a_{21}*b_{12} + a_{22}*b_{22} + a_{23}*b_{32} & a_{31}*b_{12} + a_{32}*b_{22} + a_{33}*b_{32}\\
                        a_{11}*b_{13} + a_{12}*b_{23} + a_{13}*b_{33} & a_{21}*b_{13} + a_{22}*b_{23} + a_{23}*b_{33} & a_{31}*b_{13} + a_{32}*b_{23} + a_{33}*b_{33}\\
                      \end{bmatrix}
                  \] 
      \end{enumerate}
\end{itemize}
% Rest of th\
\end{document}
