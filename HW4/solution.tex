\documentclass[12pt,letterpaper]{article}
\usepackage{fullpage}
\usepackage[top=2cm, bottom=4.5cm, left=2.5cm, right=2.5cm]{geometry}
\usepackage{amsmath,amsthm,amsfonts,amssymb,amscd}
\usepackage{lastpage}
\usepackage{enumerate}
\usepackage{fancyhdr}
\usepackage{mathrsfs}
\usepackage{xcolor}
\usepackage{graphicx}
\usepackage{listings}
\usepackage{hyperref}
\usepackage{booktabs}
\usepackage{array}
\usepackage{xcolor}
\usepackage{colortbl}

\hypersetup{%
  colorlinks=true,
  linkcolor=blue,
  linkbordercolor={0 0 1}
}
 
\renewcommand\lstlistingname{Algorithm}
\renewcommand\lstlistlistingname{Algorithms}
\def\lstlistingautorefname{Alg.}

\lstdefinestyle{Python}{
    language        = Python,
    frame           = lines, 
    basicstyle      = \footnotesize,
    keywordstyle    = \color{blue},
    stringstyle     = \color{green},
    commentstyle    = \color{red}\ttfamily
}

\setlength{\parindent}{0.0in}
\setlength{\parskip}{0.05in}

% Edit these as appropriate
\newcommand\course{CSCI 2824}
\newcommand\hwnumber{4}                  % <-- homework number
\newcommand\NetIDa{Coovi Meha}           % <-- NetID of person #1
\newcommand\NetIDb{SID:107069433}           % <-- NetID of person #2 (Comment this line out for problem sets)

\pagestyle{fancyplain}
\headheight 35pt
\lhead{\NetIDa}
\lhead{\NetIDa\\\NetIDb}                 % <-- Comment this line out for problem sets (make sure you are person #1)
\chead{\textbf{\Large Homework \hwnumber}}
\rhead{\course \\ \today}
\lfoot{}
\cfoot{}
\rfoot{\small\thepage}
\headsep 1.5em

\begin{document}

\section*{Problem 1}
Use rules of inference to show that:\\
\begin{tabular}{  m{5cm}  m{8cm}}\\
1. \({\forall}x(P(x) \lor Q(x)\))  &  promise\\
\\
2. \(P(a) \lor Q(a)\)  &  Universal instantiation from (1)\\
\\
3. \({\forall}x(\neg Q(x) \lor S(x)\))  &  promise \\
\\
4. \(\neg Q(a) \lor S(a)\) &  Universal instantiation from (3) \\
\\
5. \({\forall}x( R(x) \rightarrow \neg S(x)\))  &  promise\\
\\
6. \( R(a) \rightarrow \neg S(a)\)  &  Universal instantiation (5)\\
\\
7.\({\exists}x\neg P(x)\)  &  promise\\
\\
8.\(\neg P(a)\)  &  Existential instantiation from (7)\\
\\
9. \(P(a) \lor S(a)\) &  Resolution (2 and 4) \\
\\
10. \(S(a)\) &  Disjunctive syllogism (8 and 9) \\
\\
11. \(R(a)\) &  Modus tollens (5 and 10) \\
\\
\(\dot{.\hspace{.095in}.}\hspace{.1in}{\exists}x\neg R(x)\)  &  Existential generalization\\
\end{tabular}
\section*{Problem 2}
Prove or disprove the following claims.
\begin{enumerate}
\item  If the average of a1, a2, ..., an is some number a, then at least one of the real numbers a1, a2, ..., an
must be greater than or equal to a.
Let prove the statement using Proofs by Contradiction. That is  let assume
one of the real numbers a1, a2, ..., an
that all a(s) must be less than a.. Thus\\\\
\(\frac{(a_1+a_2+a_3+...+a_n)}{n}=a_{Avg}\) (1)\\ 
also \\\\
\(a_1<a_{Avg}\\ a_2< a_{Avg} \\ a_3 <a_{Avg} \\ ...\\...\\a_n <a_{Avg}\) (2)\\
From the sum of hypothesis (2) on the left and right of the inequality we have, \\\\
\(a_1+a_2+a_3+...+a_n<na_{avg}\) (3)\\\\
\(a_1+a_2+a_3+...+a_n<n(\frac{(a_1+a_2+a_3+...+a_n)}{n})\) from (1) and (3) then\\\\
\(a_1+a_2+a_3+...+a_n<a_1+a_2+a_3+...+a_n\) (4)\\\
The result (4) is contradictory. Thus the hypothesis " If the average of a1, a2, ..., an is some number a, then at least one of the real numbers a1, a2, ..., an
must be greater than or equal to a" is true
\item If n is an integer and 3n + 2 is even, then n is even.\\
Let use direct prove. Let k be an integer, assume that n is even, thus:\\
\(n=2k\) so\\
\(6k+2=2(3k+1)\)\\
because  k is an integer, 3k+1 must be an integer and let that integer be c,then
3n+2=2c. Therefore, we can conclude that the hypothesis is true
\item If the lengths of two sides of a triangle are irrational, then the third side must be irrational also.\\
Direct prove\\
Let a, b, c be the a number, thus \(\sqrt{a}\), \(\sqrt{b}\), \(\sqrt{c}\) are irrational.
\\ To prove this hypothesis, we just need one case that is false. So, assume, we have
a right triangle, thus\\\\
\((\sqrt{a})^2\)=\((\sqrt{b})^2\) + \((\sqrt{c})^2\)\\\\
a=\(\sqrt{b+c}\)\\\\
Let b+c=\(h^2\) with h be any integer, thus\\
a=h (which is rational)\\
Therefore the hypothesis is false
\end{enumerate}
\section*{Problem 3}
The divisibility rule by 3 is a rule that a number N is divisible by 3\\
N = 100a + 10b + c and \\
a + b + c = 3n\\
N= (99+1)a +(9+1)b +c\\
N=99a + 9b + a +b + c\\
N= 3n + 3(33a +3b)\\
\(\frac{N}{3}=(n+33a+b)\) with a b and c integer thus n is an integer\\
The hypothesis is true

\section*{Problem 4}
Prove the following:
\begin{enumerate}
  \item Prove that n is even if and only if \(n^2\) - 6n + 5 is odd.\\
  Let assume n is even, thus if k is an integer:\\
  n=2k\\
  \(n^2\) - 6n + 5=\(4k^2+6k+5\)\\
  \(n^2\) - 6n + 5=\(4k^2+6k+4+1\)\\
  \(n^2\) - 6n + 5=\(2(2k^2+3k+2)+1\) with k an integer, let an integer m =\(2k^2+3k+2\), thus \\
  \(n^2\) - 6n + 5= 2m+1 (odd)\\
  Therefore, the hypothesis is true.
  \item Prove that if \(2n^2 + 3n + 1\) is even, then n is odd.\\
  Let assume n is even. With k be a integer:\\
  n=2k\\
  \(2n^2 + 3n + 1\)=\(4k^2 + 6k + 1\)\\
  \(2n^2 + 3n + 1\)=\(2(2k^2 + 3k) + 1\) let be m is an integer with m=\(2k^2 + 3k\), thus:\\
  \(2n^2 + 3n + 1\)=\(2m+ 1\) (odd), contradictory.\\
  Therefore the hypothesis is true
\end{enumerate}
\section*{Problem 5}
Use proof by cases to prove that x + \(\mid x - 8\mid \geq\) 8 for all real numbers x. 
\begin{enumerate}
  \item Case 1: \( x - 8\leq\) 0 \\
  x + \(\mid x - 8\mid \) = x  - \( x + 8\)\\
  x + \(\mid x - 8\mid \) = 8 Therefore, for all x less or equal to 0 the hypothesis is true
  \item Case 2: \( x - 8>\) 0 \\
  x + \(\mid x - 8\mid \) = x  + \( x - 8\)\\
  x + \(\mid x - 8\mid \) = 2x - 8 \\
  2x - 8 \(>\) 8\\
  2x \(>\) 16 \\
  x \(>\) 8 \\
  x - 8 \(>\) 0  therefore, for all x greater than 0 the hypothesis is true, thus "x + \(\mid x - 8\mid \geq\) 8 for all real numbers x"\\
\end{enumerate}

% Rest of the work...
\end{document}
