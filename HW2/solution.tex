\documentclass[12pt,letterpaper]{article}
\usepackage{fullpage}
\usepackage[top=2cm, bottom=4.5cm, left=2.5cm, right=2.5cm]{geometry}
\usepackage{amsmath,amsthm,amsfonts,amssymb,amscd}
\usepackage{lastpage}
\usepackage{enumerate}
\usepackage{fancyhdr}
\usepackage{mathrsfs}
\usepackage{xcolor}
\usepackage{graphicx}
\usepackage{listings}
\usepackage{hyperref}
\usepackage{booktabs}
\usepackage{array}
\usepackage{xcolor}
\usepackage{colortbl}

\hypersetup{%
  colorlinks=true,
  linkcolor=blue,
  linkbordercolor={0 0 1}
} 
\renewcommand\lstlistingname{Algorithm}
\renewcommand\lstlistlistingname{Algorithms}
\def\lstlistingautorefname{Alg.}

\lstdefinestyle{Python}{
    language        = Python,
    frame           = lines, 
    basicstyle      = \footnotesize,
    keywordstyle    = \color{blue},
    stringstyle     = \color{green},
    commentstyle    = \color{red}\ttfamily
}

\setlength{\parindent}{0.0in}
\setlength{\parskip}{0.05in}

% Edit these as appropriate
\newcommand\course{CSCI 2824}
\newcommand\hwnumber{2}                  % <-- homework number
\newcommand\NetIDa{Coovi Meha}           % <-- NetID of person #1
\newcommand\NetIDb{SID:107069433}           % <-- NetID of person #2 (Comment this line out for problem sets)

\pagestyle{fancyplain}
\headheight 35pt
\lhead{\NetIDa}
\lhead{\NetIDa\\\NetIDb}                 % <-- Comment this line out for problem sets (make sure you are person #1)
\chead{\textbf{\Large Homework \hwnumber}}
\rhead{\course \\ \today}
\lfoot{}
\cfoot{}
\rfoot{\small\thepage}
\headsep 1.5em

\begin{document}

\section*{Problem 1}
Finding the nature of the inhabitant of the island.

A: Aramis is a Knight
B: Berthand is a Knight
C: Charleston is a Knight\\
They say:\\
Aramis: \({\neg}\) B\\
Berthand: \({\neg}\) C \({\oplus}\) B\\
Charleston: A \({\land}\) \({\neg}\) B\\
lets D:\({\neg}\) B \({\Leftrightarrow}\) \({\neg}\) C \({\oplus}\) B\\
E: \({\neg}\) B \({\Leftrightarrow}\) A \({\land}\) \({\neg}\) B\\
G: \({\neg}\) C \({\oplus}\) B \({\Leftrightarrow}\) A \({\land}\) \({\neg}\) B\\\\
\begin{tabular}{ | m{1cm} | m{1cm}| m{1cm} | m{2em} | m{1cm}| m{4em} | m{5em} | m{1cm}| m{1cm} |  m{1em} | m{2cm}|}
  \hline
  \rowcolor{gray}
  A & B & C & \({\neg}\) B & \({\neg}\) C & \({\neg}\) C \({\oplus}\) B & A \({\land}\) \({\neg}\) B
  &  D & E & G & D\({\land}\)E\({\land}\)G \\
  \hline
  T & T & T & F & F & T & F & F & T & F & F\\
  \hline
  T & F & F & T & T & T & T & T & T & T & T\\
  \hline
  T & T & F & F & T & F & F & T & T & T & T\\
  \hline
  T & F & T & T & F & F & T & F & T & F & F\\
  \hline
  F & T & T & F & F & T & F & F & T & F & F\\
  \hline
  F & F & F & T & T & T & F & T & F & F & F\\
  \hline
  \rowcolor{yellow}
  F & T & F & F & T & F & F & T & T & T & T\\
  \hline
  F & F & T & T & F & F & F & F & F & T & F\\
  \hline
\end{tabular}\\
Bertrand is a Night, Charleston is Knave and Aramis is a Knave
\section*{Problem 2}
\begin{enumerate}
  \item Using the truth table to demonstrate that ((p\({\rightarrow}\)q)\({\land}\)(q\({\rightarrow}\)r)\({\rightarrow}\) (p\({\rightarrow}\)r)
  is a tautology
  \begin{tabular}{ | m{1cm} | m{1cm}| m{1cm} | m{1cm} | m{1cm}| m{3cm} | m{1cm} | m{5cm}|}
    \hline
    \rowcolor{gray}
    p & q & r &p\({\rightarrow}\)q& q\({\rightarrow}\)r&(p\({\rightarrow}\)q)\({\land}\)(q\({\rightarrow}\)r)&p\({\rightarrow}\)r& ((p\({\rightarrow}\)q)\({\land}\)(q\({\rightarrow}\)r)\({\rightarrow}\) (p\({\rightarrow}\)r)\\
    \hline
    T & T & T & T & T & T & T & T\\
    \hline
    T & F & F & F & T & F & F & T\\
    \hline
    T & T & F & T & F & F & F & T\\
    \hline
    T & F & T & F & T & F & T & T\\
    \hline
    F & T & T & T & T & T & T & T\\
    \hline
    F & F & F & T & T & T & T & T \\
    \hline
    F & T & F & T & F & F & T & T\\
    \hline
    F & F & T & T & T & T & T & T\\
    \hline
  \end{tabular}\\\\
  The statement is a tautology.
  \item Using chain of logical equivalent.\\
  ((p\({\rightarrow}\)q)\({\land}\)(q\({\rightarrow}\)r))\({\rightarrow}\) (p\({\rightarrow}\)r) \({\equiv}\)
  \({\neg}\)((\({\neg}\)p\({\lor}\)q)\({\land}\)(\({\neg}\)q\({\lor}\)r))\({\lor}\) (\({\neg}\)p\({\lor}\)r)\\
  \({\equiv}\)\({\neg}\)((\({\neg}\)p\({\lor}\)q)\({\land}\)(\({\neg}\)q\({\lor}\)r))\({\lor}\) (\({\neg}\)p\({\lor}\)r)\\
  \({\equiv}\)((p\({\land}\)\({\neg}\)q)\({\land}\)(q\({\land}\)\({\neg}\)r))\({\lor}\) (\({\neg}\)p\({\lor}\)r)\\
  \({\equiv}\)((q\({\land}\)\({\neg}\)q)\({\land}\)(p\({\land}\)\({\neg}\)r))\({\lor}\) (\({\neg}\)p\({\lor}\)r)\\
  \({\equiv}\)((p\({\land}\)\({\neg}\)r))\({\lor}\) (\({\neg}\)p\({\lor}\)r)\\
  \item Showing that (p\({\rightarrow}\)q) \({\rightarrow}\) r and p\({\rightarrow}\)(q \({\rightarrow}\) r) are not logical equivalent.\\
  \begin{tabular}{ | m{1cm} | m{1cm}| m{1cm} | m{1cm} | m{1cm}| m{3cm} | m{3cm} |}
    \hline
    \rowcolor{gray}
    p & q & r &p\({\rightarrow}\)q& q\({\rightarrow}\)r&(p\({\rightarrow}\)q) \({\rightarrow}\) r & p\({\rightarrow}\)(q \({\rightarrow}\) r)\\
    \hline
    T & T & T & T & T & T & T\\
    \hline
    T & F & F & F & T & T & T\\
    \hline
    T & T & F & T & F & T & F\\
    \hline
    T & F & T & F & T & F & T\\
    \hline
    F & T & T & T & T & T & T\\
    \hline
    F & F & F & T & T & T & T\\
    \hline
    F & T & F & T & F & T & T\\
    \hline
    F & F & T & T & T & T & T\\
    \hline
  \end{tabular}\\\\
  The last two columns contain contradictory statements, thus the two logical  statements are not equivalent.
\end{enumerate}
\section*{Problem 3}
\begin{enumerate}
  \item \({\forall}xp(x)\) \({\equiv} P(5) {\land} P(6) {\land} P(7) {\land} P(8)\)
  \item \({\neg}{\exists}xp(x)\) \({\equiv} {\neg}P(5) {\lor} {\neg}P(6) {\lor} {\neg}P(7) {\lor} {\neg}P(8)\)
  \item \({\neg}{\forall}xp(x)\)\({\equiv} {\neg}P(5) {\land} {\neg}P(6) {\land} {\neg}P(7) {\land} {\neg}P(8)\)
\end{enumerate}
\section*{Problem 4}
\begin{tabular}{  m{1cm} | m{1cm}}
  163 & 1\\
  54 & 0\\
  18 & 0\\
  6 & 0\\
  2 & 2\\
\end{tabular} 
163 in base 3 is \((20001)_3\)

\section*{Problem 5}
\begin{enumerate}
  \item Translating each of the group’s travel requirements from English into a proposition.
  \begin{enumerate}
    \item Shaggy: V \({\lor}\) \({\neg}\) S;
    \item Velma: P \({\rightarrow}\) \({\neg}\) V
    \item Daphne: B \({\leftrightarrow}\) (L \({\land}\) P)
    \item Fred: \({\neg}\) P
    \item Scooby: \({\exists}xT(x)\) \({\equiv}\) P \({\lor}\) V \({\lor}\) S \({\lor}\) B
  \end{enumerate}
  \item Satisfaction of the gangs proposition. Lets Sat be the logical the gangs go on vacation. For Sat to be True:
  (V \({\lor}\) \({\neg}\) S)  \({\land}\) (P \({\rightarrow}\) \({\neg}\) V) \({\land}\) (B \({\leftrightarrow}\) (L \({\land}\) P)) \({\land}\) (\({\neg}\) P) \({\land}\)
(P \({\lor}\) V \({\lor}\) S \({\lor}\) B) must be true.
  \item The team travel whish is possible. If V is true proposition a) will be true no 
  matter the values of S. For b) to be true, P must be false. The statement
  c) is true only if B is false no matter the outcome of L. Because P was false, the statement d) must be true.Finally as one of the valuein each
  is true, statement e) is true.\\
  -Venice\\
  -Shanghai\\
  \item Park W in west Africa.
\end{enumerate}
% Rest of the work...
\end{document}
