\documentclass[12pt,letterpaper]{article}
\usepackage{fullpage}
\usepackage[top=2cm, bottom=4.5cm, left=2.5cm, right=2.5cm]{geometry}
\usepackage{amsmath,amsthm,amsfonts,amssymb,amscd}
\usepackage{lastpage}
\usepackage{enumerate}
\usepackage{fancyhdr}
\usepackage{mathrsfs}
\usepackage{xcolor}
\usepackage{graphicx}
\usepackage{listings}
\usepackage{hyperref}
\usepackage{booktabs}
\usepackage{array}
\usepackage{xcolor}
\usepackage{colortbl}

\hypersetup{%
  colorlinks=true,
  linkcolor=blue,
  linkbordercolor={0 0 1}
}
\renewcommand\lstlistingname{Algorithm}
\renewcommand\lstlistlistingname{Algorithms}
\def\lstlistingautorefname{Alg.}

\lstdefinestyle{Python}{
    language        = Python,
    frame           = lines, 
    basicstyle      = \footnotesize,
    keywordstyle    = \color{blue},
    stringstyle     = \color{green},
    commentstyle    = \color{red}\ttfamily
}

\setlength{\parindent}{0.0in}
\setlength{\parskip}{0.05in}

% Edit these as appropriate
\newcommand\course{CSCI 2824}
\newcommand\hwnumber{6}                  % <-- homework number
\newcommand\NetIDa{Coovi Meha}           % <-- NetID of person #1
\newcommand\NetIDb{SID:107069433}           % <-- NetID of person #2 (Comment this line out for problem sets)

\pagestyle{fancyplain}
\headheight 35pt
\lhead{\NetIDa}
\lhead{\NetIDa\\\NetIDb}                 % <-- Comment this line out for problem sets (make sure you are person #1)
\chead{\textbf{\Large Homework \hwnumber}}
\rhead{\course \\ \today}
\lfoot{}
\cfoot{}
\rfoot{\small\thepage}
\headsep 1.5em

\begin{document}

\section*{Problem 1}
\begin{itemize}
  \item Suppose P, Q, and R are non-empty sets. Prove that Px(Q \(\cap\) R) = (PxQ) \(\cap\) (PxR) by showing
        that each side of this equation must be a subset of the other side, and concluding that the two
        sides must therefore be equal.\\
        Px(Q \(\cap\) R) \(\subseteq\) (PxQ) \(\cap\) (PxR)\\
        let (x,y) \(\in\) Px(Q \(\cap\) R)\\
        x \(\in\)  P  and y \(\in\) (Q \(\cap\) R)\\
        x \(\in\)  P  and y \(\in\) R  and Q \(\in\) R\\
        (x,y) \(\in\)  PxR and  (x,y) \(\in\) P x Q\\
        (x,y) \(\in\) (PxQ) \(\cap\) (PxR)\\\\
        (PxQ) \(\cap\) (PxR) \(\subseteq\) Px(Q \(\cap\) R)\\
        let (x,y) \(\in\)  (PxQ) \(\cap\) (PxR)\\
        (x,y)\(\in\) PxQ and (x,y)\(\in\) PxR\\
        x \(\in\) P,  y \(\in\) Q  and x \(\in\) P, y \(\in\) R\\
        x \(\in\) P and x \(\in\) P, y \(\in\) R\\
        x \(\in\) P and (x,y) \(\in\) PxR\\
        (x,y) \(\in\) P \(\cap\) PxR\\
        Thus  Px(Q \(\cap\) R) = (PxQ) \(\cap\) (PxR)
  \item Suppose that P, Q, and R are non-empty sets. Prove that Px(Q \(\cap\) R) = (PxQ) \(\cap\) (PxR) by
        using set builder notation and set identities and definitions.\\
        let (x,y) \(\in\) Px(Q \(\cap\) R), this is equivalent to\\
        Px(Q \(\cap\) R) = {(x,y) : x \(\in\) P, (x,y) (Q \(\cap\) R)}\\
        Px(Q \(\cap\) R) = {(x,y) : x \(\in\) P, x \(\in\) Q and y \(\in\) R}\\
        Px(Q \(\cap\) R) = {(x,y) : x \(\in\) P, x \(\in\) Q and x \(\in\) P, y \(\in\) R}\\
        Px(Q \(\cap\) R) = {(x,y) : (x,y) \(\in\) PxQ and (x,y) \(\in\) PxR}\\
        Px(Q \(\cap\) R) = {(x,y) : (x,y) \(\in\) PxQ and (x,y) \(\cap\) PxR}\\
        Thus  Px(Q \(\cap\) R) = (PxQ) \(\cap\) (PxR)
  \item Let U be the set of all integers. Let E be the set of all even integers, D the set of all odd integers,
        P the set of positive integers, and N the set of all negative integers. Find the following sets.\\
        i. Set of all even Negative integers
        ii. Set of all integers (positives and negative integers)
        iii. Set of all odd integers
        iv. Set all non integers
\end{itemize}
\section*{Problem 2}
\begin{itemize}
  \item Give an example of two uncountable sets A and B with a nonempty intersection, such that A − B
        is\\
        i. A set with size 0 such as A and B are equal\\
        ii.  A is set of all real number and B is set of all real number except negative integer. so A-B is
        set of all positive integers.\\
        iii.  A is set of all real number, and B is set all positive real number. Thus A-B is set of all
        real negative number.
  \item Use the Cantor diagonalization argument to prove that the number of real numbers in the interval
          [3, 4] is uncountable.\\
        Suppose that we can list all the number in the [3,4]\\
        3.100110111\\
        3.100101011\\
        3.100011011\\
        3.01110101\\
        ....\\
        ....\\
        let m=b1b2b3b4 and\\
        bi={0 if aii different than 1, 1 if aii=0}\\
        m=0110....\\
  \item Use a proof by contradiction to show that the set of irrational numbers that lie in the interval
          [3, 4] is uncountable. (You can use the fact that the set of rational numbers (Q) is countable and
        the set of reals (R) is uncountable). Show all work.\\
        Assume that the set all irrational in [3,4] is countable\\
        We know that the set of all rational in [3,4] is countable and the set of reals (R) is uncountable.\\
        The subtraction of (Q) and (R) must then be uncountable which is the set
        of all irrational number in [3,4]. This is a contradiction.
\end{itemize}
\section*{Problem 3}
\begin{itemize}
  \item Find a closed form for the recurrence relation: \(a_n\) = 2\(a_{n-1}\) - 2, \(a_0\) = -1\\
        \(a_0\) = -1\\
        \(a_1\) = -4 =-4 -2 +2 \\
        \(a_2\) = -10= -8 -4 +2\\
        \(a_3\) = -22 = -16 -8 +2\\
        \(a_n\) = -\(2^{n+1}\) - \(2^n\) + 2
  \item Find a closed form for the recurrence relation: \(a_n\) = (n + 2)\(a_{n-1}\), \(a_0\) = 3\\
        \(a_0\) = 3\\
        \(a_1\) = (1 + 2)\(a_{0}\) =9\\
        \(a_2\) = (2 + 2)\(a_{1}\) =36\\
        \(a_3\) = (3 + 2)\(a_{2}\) =180\\
        \(a_4\) = (4 + 2)\(a_{3}\) =1080\\
        \(a_n\) =\(\frac{3}{2}\) (n+2)!
  \item  Show that \(a_{n}\) = \(5(-1)^n\) -n + 2 is a solution of the recurrence relation \(a_n\) = \(a_{n-1}\) + 2\(a_{n-2}\) + 2n - 9.\\
        let \(a_n\) = \(a_{n-1}\) + 2\(a_{n-2}\) + 2n - 9\\
        if \(a_{n}\) = \(5(-1)^n\) -n + 2, then \(a_{n-1}\) = \(5(-1)^{n-1}\) -n + 3 and \(a_{n-2}\) = \(5(-1)^{n-2}\) -n + 4\\
        \(a_n\) = \(a_{n-1}\) + 2\(a_{n-2}\) + 2n - 9 become\\
        \(a_n\) = \(5(-1)^{n-1}\) -n + 3 + 2*\(5(-1)^{n-2}\) -2n + 8 + 2n - 9\\
        \(a_n\) = \(5(-1)^{n}*(-1)^{-1}\) + 2*\(5(-1)^{n}*(-1)^{-2}\) + 2 - n\\
        \(a_{n}\) = \(5(-1)^n\) -n + 2\\
        \(a_{n}\)=\(a_{n}\)
\end{itemize}
\section*{Problem 4}
\begin{itemize}
  \item Consider the function f : Z x Z \(\rightarrow \) Z where f(m, n) = 2m - n. Is this function onto?\\\\
  The function is onto because for every 2m-n there is an integers
  m and n for which 2m-n is integers
  \item Consider the function f : Z x Z \(\rightarrow \) Z where f(m, n) = \(m^2\) - \(n^2\). Is this function onto?\\\\
  The function is not onto because  \(m^2\) or \(n^2\) can not be satisfied when 
  negative that is there is not inter for which \(m^2\) or \(n^2\) are negatives.
  \item Define the set C = the set of all residents of Colorado. Define in words a function f : C \(\rightarrow \) Z. Is
  your function one-to-one? Is it onto? Be sure that the f you defined is indeed a function. Be
  creative and have fun!\\\\
  let x be number of job an individual has\\
  let y how many hour an individual works per week\\
  The function f(x,y)= xy is not onto or one to one because for every total
  hour of work xy we can not find a resident with x and y satisfied. For example no individual work 1000 hours per week.
  \item  Again, define the set C = the set of all residents of Colorado. Define in words a function f : C \(\rightarrow \) Z.
  However this time, make sure that your function is one-to-one. (Make sure to give a different
  example from part (c)).\\\\
  Let f(x)= m with m be how much each resident spend at grocery store per month.
  f is  one-to-one because the amount the individual x spend 
  in one week will have a unique value in z 
\end{itemize}

% Rest of the work...
\end{document}
